\documentclass{beamer}

\usepackage[frenchb]{babel}
\usepackage[T1]{fontenc}
\usepackage[utf8]{inputenc}
\usepackage{xifthen}

\usetheme{Warsaw}

\title{Cassandra}
\author{Michel Barret}
\institute{Viseo}

\newenvironment{slideTitle}[2]
{
  \ifthenelse{\isempty{#1}}{}{\section{#1}}
  \ifthenelse{\isempty{#2}}{}{\subsection{#2}}
  \begin{frame}
    \ifthenelse{\isempty{#1}}{
      \ifthenelse{\isempty{#2}}{}{\frametitle{#2}}
    }{
      \frametitle{#1}
    }
}
{ \end{frame} }
%\newcommand\slideSection[1]{#1~\textsc{#2}}

\begin{document}

\begin{frame}
  \titlepage
\end{frame}

\begin{frame}
  \tableofcontents
\end{frame}

\begin{slideTitle}{Introduction}{Contexte}
\begin{itemize}
\item initié par Facebook
\item développé par Datastax et Apache
\item utilisé entre autre par Apple
\end{itemize}
\end{slideTitle}

\begin{slideTitle}{}{Le NoSQL}
\begin{itemize}
\item pas de relations
\item pas de transaction
\item pas de trigger
\item pas de contraintes d'intégrité
\item pas de forme normale
\item du P2P
\item de larges colonnes
\end{itemize}
\end{slideTitle}

\begin{slideTitle}{Concepts}{Modélisation}
\begin{itemize}
\item keyspace~: c'est la base de données au sens MySQL/MariaDB
\item table~: comme vous connaissez déjà
\item clefs~: primaire, composite, de clustering, de partitionnement,\dots (oui)~:
% faire un joli dessin ^_^
\item index~:
\end{itemize}
\end{slideTitle}

\begin{slideTitle}{}{Les clefs}
\begin{itemize}
\item clefs~: primaire, composite, de clustering, de partitionnement,\dots (oui)~:
% faire un joli dessin ^_^
% voir s'il y a besoin de décrire l'architecture de cassandra si c'est le cas
% il faut simplifier et revenir là dessus plus tard (voir avec d'autres (Thomas & Pick))
\item index~:
\end{itemize}
\end{slideTitle}

\begin{slideTitle}{}{Types}
\begin{itemize}
\item entier, text, etc
% pas d'UDT
\item *Exercice* : Première modélisation
\end{itemize}
\end{slideTitle}

\begin{slideTitle}{}{Requêtage}
\begin{itemize}
% ceci n'est pas du SQL
\item présentation
\item *Exercice* : Créer sa/ses tables, les charger, puis les requêter
\end{itemize}
\end{slideTitle}

\begin{slideTitle}{}{Fonctionnalités avancées}
\begin{itemize}
\item User Data Type et collection
*Exercice* : ajouter un nouveau type
\item vues matérialisées : prendre en compte de nouveaux besoins
*Exercice* : mettre en place une vue materialisée
\item utilisation d'un index
\end{itemize}
\end{slideTitle}

\begin{slideTitle}{Développement}{Driver standard}
\begin{itemize}
\item requetage
  \begin{itemize}
    \item QueryFactory
    \item requêtes nommées (par anotation)
  \end{itemize}
\item gestion du point d'accès
\end{itemize}
\end{slideTitle}

\begin{slideTitle}{Cluster}{Modélisation}
\begin{itemize}
\item clef de clustering
\item transaction légère => mise en évidence
\item topologie réseau
\end{itemize}
\end{slideTitle}

\begin{slideTitle}{}{Requêtage}
\begin{itemize}
\item ONE/QUORUM/ALL
\end{itemize}
\end{slideTitle}

\begin{slideTitle}{}{Exploitation ?}
\begin{itemize}
\item pas elastique :(
\end{itemize}
\end{slideTitle}

\end{document}
